\RequirePackage[l2tabu, orthodox]{nag}
\documentclass[a4paper,draft,12pt,oneside,article,table]{memoir}

%% Geometry
\isopage[10]
\setlength{\topskip}{1.6\topskip} % for \sloppybuttom
\checkandfixthelayout
\sloppybottom

%% Typography
\usepackage{polyglossia,microtype,hyperref,amsmath,unicode-math,xcolor,natbib}
\definecolor{zen-red}{HTML}{B23333}    \definecolor{zen-orange}{HTML}{E57A33}
\definecolor{zen-yellow}{HTML}{F0DFAF} \definecolor{zen-green}{HTML}{5F7F5F}
\definecolor{zen-cyan}{HTML}{93E0E3}   \definecolor{zen-blue}{HTML}{336CB2}
\setdefaultlanguage{english} %polyglossia

\hypersetup{colorlinks=true,linkcolor=zen-red,citecolor=zen-green,urlcolor=zen-orange} % hyperref
\microtypesetup{final,verbose=silent} 
\setmainfont[Ligatures=TeX,Numbers=OldStyle]{Arno Pro} % fontspec
\setmonofont[Scale=MatchLowercase]{DejaVuSansMono} % fontspec

\unimathsetup{math-style=ISO,bold-style=ISO} % unicode-math
\setmathfont[Scale=MatchLowercase]{Cambria Math} % unicode-math


%% Titlepage
\setlength{\droptitle}{-3em}
\pretitle{\LARGE\par} \posttitle{\vskip 0.5em}
\newcommand{\supertitle}[1]{\gdef\suP{#1}}
\renewcommand{\maketitlehooka}{\ifx\suP\undefined\begin{center}\else\begin{center} {\scshape\suP}\fi}
\newcommand{\subtitle}[1]{\gdef\suB{#1}}
\renewcommand{\maketitlehookb}{\ifx\suB\undefined \end{center}\else\par {\large\scshape\suB}\par\end{center}\fi}

%% Header
\newcommand{\stunum}[1]{\gdef\stuN{#1}}
\copypagestyle{articlehead}{plain}
\makeoddhead{articlehead}{\color{gray}\theauthor\ifx\stuN\undefined\else\ifx\stuN\empty\else~(\stuN)\fi\fi}{}{\color{gray}\thedate}
\pagestyle{articlehead}

%% Grapics
\usepackage{tikz,pgfplots,tikz-timing}
\usetikzlibrary{mindmap,arrows,positioning,shapes}

%% TÅGEKAMMERET
%\usepackage{tket,tkvc}
%\newfontface\bbface[Scale=0.87]{TeX Gyre Termes Math} \TKsetup{C = {\bbface\kern-0.1exℂ}} % fontspec,tket

%% resten
\usepackage{threeparttable,siunitx,pdfpages,algpseudocode,algorithm}
\sisetup{per-mode=symbol}
\makeatletter \renewcommand{\ALG@name}{Algoritme}\makeatother
\usepackage{minted} % requires minted > 2.0-alpha2
\usemintedstyle{tango}

\newcommand{\srcpath}{../ex09/src/main/java/ddist}
\newcommand{\inmnt}[3]{\noindent\texttt{\color{gray}File: #3}\vspace{-1em}\inputminted[tabsize=4,firstline=#1,firstnumber=#1,lastline=#2,linenos]{java}{\srcpath/#3}}
\newcommand{\mil}[1]{\mintinline{java}{#1}}

%% Help
\usepackage{lipsum}
\usepackage[margin,draft]{fixme} \fxusetheme{color}

\begin{document}
\supertitle{Distributed Systems}
\title{Exercise 9}
%\subtitle{}
\author{Richard~Möhn~\small{(201311231)} \and Mathias~Dannesbo~\small{(201206106)}}
%\stunum{201311231, 201206106}
\date{\today}
\maketitle

\chapter{Introduction}
We used pairprogramming for all the code and ``pairreporting'' for the report, so we share the workload at 50\% each.

\chapter{Code overview}
We need to preform actions when the \mil{Listen}, \mil{Connect} and \mil{Disconnect} items are clicked in the menus. The items represent \mil{javax.swing.Actions} whose \mil{actionPerformed()} methods with our code.

\section{Listen}

The \mil{Listen} is parsing the server information from the \textsc{gui} and putting a new Runnable in the \textsc{awt} system \mil{EventQueue} whitch start up a new \mil{ServerSocket} for listening for another editor to connect as a client. Then it calls \mil{startCommunication} which is described in section \ref{sec:com}. Finally it set the title of the editor.

\inmnt{139}{193}{DistributedTextEditor.java}

\section{Connect}
The \mil{Connect} first clears the textareas, then parses the server information from the \textsc{gui} and connect to the corresponding socket. When the \textsc{tcp/ip} connection is etablished, it calls the \mil{startCommunication} which is described in section \ref{sec:com}. Finally it set the title of the editor.

\inmnt{195}{228}{DistributedTextEditor.java}

\section{Communication between editors}
\label{sec:com}
\tikzset{
  global/.style={draw=black,line width=1pt, inner sep=.5em, minimum height=3em, minimum width=11em, text centered},
  c/.style={global, cloud, aspect=1.5, cloud puffs=15,},
  n/.style={global, rectangle},
  r/.style={n, fill=red!20},
  g/.style={n, fill=green!20},
  b/.style={n, fill=blue!20},
  y/.style={n, fill=yellow!20},
  l/.style={>=latex',line width=1.5pt},
  sa/.style={l,->,shorten >=1pt},
}

\begin{figure}[hbp]
  \centering
  \caption{\mil{MyTestEvent} path through the system.}
  \begin{tikzpicture}[node distance=1cm, auto, scale=0.8]
    \node[c] (net) {Network};
    \node[b, above=2em of net] (esen) {EventSender};
    \node[g, above=2em of esen] (oqueue) {Outbound queue};
    \node[r, above=2em of oqueue] (dec2) {DocumentEventCapturer};
    \node[b, below=2em of net] (erec) {EventReceiver};
    \node[g, below=2em of erec] (iqueue) {Inbound queue};
    \node[r, below=2em of iqueue] (erep2) {EventReplayer};
    \node[r, left=2em of dec2] (dec) {DocumentEventCapturer};
    \node[g, left=2em of net] (queue) {Queue};
    \node[r, left=2em of erep2] (erep) {EventReplayer};

    \node[above=1.5em of dec] (bef) {Provided};
    \node[above=1.5em of dec2] (aft) {Network enabled};


    \draw[sa] (dec2.south) --  (oqueue.north);
    \draw[sa] (oqueue.south) --  (esen.north);
    \draw[sa] (esen.south) --  (net.north);
    \draw[sa] (net.south) --  (erec.north);
    \draw[sa] (erec.south) --  (iqueue.north);
    \draw[sa] (iqueue.south) --  (erep2.north);

    \draw[sa] (dec.south) --  (queue.north);
    \draw[sa] (queue.south) --  (erep.north);
  \end{tikzpicture}
  \label{fig:event}
\end{figure}
Before the editor was network-enabled, there was one queue where the information from the upper textarea was stored before it was displayed on the lower textarea. I our network-enabled editor there are effectivly two queues. Each effective queue is implemented with a queue on each side of the network connection.

When a \mil{MyTextEvent} is send over the network it leaves the queue on the sending side and enters the queue on the reciving side. 


Witout \textsc{rmi} we cannot put object in queue on the 
is implemented in the \texttt{EventReciever.java}

\inmnt{43}{67}{DistributedTextEditor.java}

\inmnt{300}{318}{DistributedTextEditor.java}

\inmnt{1}{1000}{EventReplayer.Java}

\inmnt{30}{39}{DocumentEventCapturer.Java}

\inmnt{1}{1000}{EventReceiver.java}

\inmnt{1}{1000}{EventSender.java}

\section{Disconnect}
\inmnt{230}{240}{DistributedTextEditor.java}
\inmnt{1}{1000}{DisconnectEvent.java}

%\inmnt{1}{10}{MyTextEvent.java}
%\inmnt{1}{10}{TextInsertEvent.java}
%\inmnt{1}{10}{TextRemoveEvent.java}

\chapter{Conclusion}
\fxerror{List of concepts}

%\clearpage
% \listoftables
% \listoffigures
% \listoflistings
%\nocite{*}
%\bibliographystyle{dlfltxbbibtex} \Bibliography{bib}
%\clearpage \appendix

\end{document}

%%% Local Variables:
%%% coding: utf-8
%%% mode: latex
%%% TeX-engine: xetex
%%% End: