\documentclass[xcolor=svgnames]{beamer}

\usepackage{xltxtra}
\usepackage{polyglossia}
\setdefaultlanguage{german}
\setsansfont{Universalis ADF Std}
\setmonofont[Scale=MatchLowercase]{DejaVuSansMono}
\usepackage{textcomp}

\usepackage{pstricks}

\usepackage[german=guillemets]{csquotes}
\usepackage[backend=biber, style=ieee]{biblatex}
\addbibresource{../literatur.bib}

\mode<presentation>
{
  \usetheme{Madrid}
  \usecolortheme{dolphin}
  \usecolortheme{rose}
  \setbeamertemplate{footline}{\hfill\insertpagenumber~}

  \setbeamercovered{transparent}
}

\newcommand*{\beffont}{\ttfamily}
\usepackage[final]{listings}
\lstloadlanguages{C,[5.1]Lua}
\lstdefinestyle{default}{
    captionpos=b,
    escapeinside={--+}{+--},
    tabsize=4,
    showstringspaces=false,
    breaklines=true,
    basicstyle=\beffont,
    rulecolor=\color{gray},
    commentstyle=\color{DarkBlue},
    stringstyle=\color{DarkRed},
    keywordstyle=[1]\color{DarkGreen},
    keywordstyle=[2]\color{DarkSlateGrey},
}
\lstdefinestyle{clisting}{
    escapeinside={/*+}{+*/},
    language=C,
    basicstyle=\beffont,
    breaklines=false,
}
\lstdefinestyle{lualisting}{
    escapeinside={--+}{+--},
    language=[5.1]Lua,
}
\lstset{style=default}

\usepackage[
    copy-decimal-marker,
    exponent-product=\cdot,
    binary-units
]{siunitx}
\newcommand*{\uB}[1]{\SI{#1}{\byte}}
\newcommand*{\ukB}[1]{\SI{#1}{\kilo\byte}}
\newcommand*{\uMB}[1]{\SI{#1}{\mega\byte}}
\newcommand*{\ubit}[1]{\SI{#1}{\bit}}
\newcommand*{\uMHz}[1]{\SI{#1}{\mega\hertz}}
\newcommand*{\uGHz}[1]{\SI{#1}{\giga\hertz}}
\newcommand*{\uMbps}[1]{\SI{#1}{\mega\bit\per\second}}

\usepackage{xcolor}

\newcommand*{\vspmin}{\vspace{-\baselineskip}}
\newcommand*{\vspplus}{\vspace{\baselineskip}}


\newcommand*{\bild}[1]{../bach/bilder/#1}


\begin{document}

\defverbatim[colored]\simpleblink{%
\lstset{style=lualisting}
\begin{lstlisting}
local ledpin = pio.PB_8

pio.pin.setdir(pio.OUTPUT, ledpin)

while true do
    pio.pin.setlow(ledpin)
    tmr.delay(tmr.SYS_TIMER, 500000)

    pio.pin.sethigh(ledpin)
    tmr.delay(tmr.SYS_TIMER, 500000)
end
\end{lstlisting}%
}

\defverbatim[colored]\narrowblink{%
\lstset{style=lualisting, xleftmargin=3em}
\begin{lstlisting}
local l = pio.PB_8
local t = 500000

pio.pin.setdir(pio.OUTPUT, l)

while true do
  pio.pin.setlow(l)
  tmr.delay(tmr.SYS_TIMER, t)

  pio.pin.sethigh(l)
  tmr.delay(tmr.SYS_TIMER, t)
end
\end{lstlisting}%
}

% - http://www.inf.fu-berlin.de/w/SE/VortragsTipps
% - http://www.inf.fu-berlin.de/w/SE/VortragBewertungskriterien
% 
% 
% What do you want them to learn?
% - was eLua und Lua sind 
% - was ich in meiner Arbeit gemacht habe
% - dass es sich lohnt, eLua weiter zu untersuchen
% 
% What is their interest in what you've got to say?
% - forschen an Sensornetzwerken und IoT -- Alternativen zur
%   Programmierung und Programmverteilung
% - haben MSB-IOT entwickelt -- tollet Jerät, wa?
% - lehren -- wollen Studenten vielleicht auch Hardwareprogrammierung
%   näherbringen
% - murksen mit Hardwarezeugs rum und wollen mal was anderes sehen
% 
% How sophisticated are they?
% - Quite sophisticated.
% - Wissenschaftler und Studenten
% 
% How much detail do they want?
% - nicht zu viel
% - Arbeit selber geht auch nicht sehr ins Detail
% - wollen grob wissen, was ich gemacht habe
% - wollen hauptsächlich wissen, was diese Sachen sind und was man damit
%   anstellen kann
% 
% Whom do you want to own the information?
% - Habe ich noch nie verstanden.
% 
% How can you motivate them to listen to you?
% - nicht schwafeln, keine Slogans -- klare Fakten, klare Gedankenketten
% - ein paar Beispiele

\title{Evaluation einer Umgebung für die Skriptsprache Lua auf dem
Sensorboard MSB-IOT}

\author{Richard Möhn}

\date{Präsentation der Bachelorarbeit, 13. Januar 2014}

\subject{Informatik}

% - Begrüßung
% - Titel ...
% - Begriffe: Sensorboard, Skriptsprache -- eher selten zusammen, denn
% Sensorboards meist mit kleiner Hardware und Skriptsprachen mit
% automatischer Speicherbereinigung und so recht hungrig
% - deswegen Erprobung
\begin{frame}
  \titlepage
\end{frame}

\begin{frame}{Überblick}
% - Zielsetzung der Arbeit -- Was wollte ich machen?
% - Durchführung -- Was habe ich gemacht?
\tableofcontents
\end{frame}

\section{Motivation}

\begin{frame}[fragile]
\frametitle{Ein Beispiel zur Motivation}

\simpleblink
\end{frame}

% Motivation:
% - LED-Blink-Skript
% - und das auf Embedded-Hardware!
% - Erstrebenswert, einen Großteil der Programme so schreiben zu können

\section{Zielsetzung}

% Zielsetzung der Arbeit:
% - Portierung
% - Einbindung zweier Treiber
% => Orientierung für Umgang mit eLua auf MSB-IOT
% => Grundlage für weitere Arbeiten

\begin{frame}{Ziel der Arbeit}
\begin{itemize}
\item Lua-Skripte auf dem MSB-IOT    
\item Zugriff auf Funk-Transceiver
\item Zugriff auf WLAN-Modul
\end{itemize}

\vspplus

\quad $\Rightarrow$ Orientierung und Arbeitsgrundlage für Folgendes
\end{frame}

\section{Grundlagen}
 
% Vorstellung des MSB-IOT
% - Hardware
% - CC1101
% - CC3000
% - Sensornetzwerke, Internet of Things

\begin{frame}{\emph{M}odular \emph{S}ensor \emph{B}oard – \emph{I}nternet \emph{o}f \emph{T}hings}

\begin{columns}
\begin{column}{0.55\textwidth}
\begin{itemize}
\item STM32F415RG, Cortex-M4, \ubit{32}, bis zu \uMHz{168}
\item \ukB{196} RAM, \uMB{1} ROM
\item CC1101: Sub-\uGHz{1}-Funk-Transceiver
\item CC3000: WLAN-Modul, Netzwerkprozessor
\item ähnlich STM32F4 Discovery-Board
\end{itemize}
\end{column}

\begin{column}{0.4\textwidth}
\includegraphics[width=\textwidth]{\bild{msbiot-draufsicht.png}}
\end{column}
\end{columns}
    
\end{frame}

% Vorstellung von Lua und eLua
% - noch mal das Programm vom Anfang
% - powerful, fast, lightweight, embeddable scripting language
% - interaktives Programmieren durch REPL
% - haben schon was gesehen: MSB-IOT führt irgendwie Lua-Skripte aus
% - Wie geht das? -- Interpreter läuft direkt auf der Hardware und hat das
%   Sagen.
% - weitere Highlights:
%     - weitere Module wie gpio und tmr oben
%     - Programme von der SD-Karte
%     - neben autorun (Bsp.) auch eLua-Shell auf dem Gerät
%     - RPC: REPL von Ferne

\begin{frame}[fragile]{Lua und eLua}
\begin{columns}[T,totalwidth=0.98\textwidth]
\begin{column}{0.45\textwidth}

\begin{block}{Lua}
\begin{itemize}
\item \enquote{powerful, fast, lightweight, embeddable}
\item erweiterbar
\end{itemize}
\end{block}

\begin{block}{eLua}
\begin{itemize}
\item direkt auf der Hardware
\item Module zum Hardwarezugriff
\item Autorun oder Shell
\end{itemize}
\end{block}
\end{column}

\begin{column}{0.35\textwidth}
\narrowblink 
\end{column}
\end{columns}
\end{frame}

\begin{frame}{Einige Bytes}
% total, verbatim: 245760
%  wlan-test.lua                 1006 bytes
%  uart-test.lua                 79 bytes
%  demoblink.lua                 826 bytes
%  blink.lua                     672 bytes
%  radio-test.lua                1887 bytes
%Total on /rom: 4470 bytes
% PMs: 78032
% ohne PMs und Treiber: 227672 => PMs: 18088

% total, compress: 243032 (243029 wenn nur Skripte)
%  wlan-test.lua                 469 bytes
%  uart-test.lua                 67 bytes
%  demoblink.lua                 305 bytes
%  blink.lua                     194 bytes
%  radio-test.lua                704 bytes
%Total on /rom: 1739 bytes

% total, compile: 244344 (244346)
%   wlan-test.lc                  804 bytes
%   uart-test.lc                  208 bytes
%   demoblink.lc                  492 bytes
%   blink.lc                      396 bytes
%   radio-test.lc                 1156 bytes
% Total on /rom: 3056 bytes

% FAQ, 2014-01-12
\begin{tabular}{lr}
eLua RAM (minimal)          &   \num{32000}     \\
eLua RAM (empfohlen)        &   \num{64000}     \\
MSB-IOT RAM                 &   \num{196000}    \\
\\
eLua ROM (minimal)          &   \num{128000}    \\
eLua ROM (empfohlen)        &   \num{256000}    \\
MSB-IOT ROM                 &   \num{1000000}   \\
\\
eLua-Portierung             &   \num{245760}    \\
%davon Lua-Interpreter (?)   &   \num{182000}    \\
davon Plattformmodule       &   \num{18088}     \\
davon Skripte               &   \num{4470}      \\
davon Skripte (komprimiert) &   \num{1156}      \\
\end{tabular}
\end{frame}
 
\section{Durchführung}

% - Portierung: Bild mit Pfeilen
% 
% - CC1101:
%     - Schichtenbild
%     - Ausschnitt aus Minimalplattformmodul
% 
% - CC3000:
%     - evtl. auch Schichtenbild
%     - Beispielskript

\begin{frame}{Ansteuerung von CC1101 und CC3000}
\begin{columns}[b]
\begin{column}{0.5\textwidth}
\centering
\begin{pspicture}(0,0)(5,5.5)
\psTextFrame(0,6)(5,7){Endbenutzer}

\only<3->{
\psline{->}(2.5,6)(2.5,5.5)
\psTextFrame(0,4.5)(5,5.5){Plattformmodul}
\psline{->}(2.5,4.5)(2.5,4)
}

\only<2->{
\psTextFrame(0,1.5)(5,4){CC1101-Treiberbibliothek}
\psline{->}(2.5,1.5)(2.5,1)
}

\psTextFrame(0,0)(5,1){CC1101}
\end{pspicture}
\end{column}

\begin{column}{0.5\textwidth}
\centering
\begin{pspicture}(0,0)(4,7)
\psTextFrame(0,6)(4,7){Endbenutzer}

\only<3->{
\psline{->}(2,6)(2,5.5)
\psTextFrame(0,4.5)(4,5.5){Plattformmodul}
\psline{->}(2,4.5)(2,4)
}

\only<2->{
\psTextFrame(0,3)(4,4){CC3000-Bibliothek}
\psline{->}(2,3)(2,2.5)

\psTextFrame(0,1.5)(4,2.5){SPI-Funktionen u. a.}
\psline{->}(2,1.5)(2,1)
}

\psTextFrame(0,0)(4,1){CC3000}
\end{pspicture}
\end{column}
\end{columns}    
\end{frame}

\begin{frame}[fragile]{Eine Funktion des CC3000-Plattformmoduls}
\lstset{style=clisting}
\begin{lstlisting}
static int msbiot_cc3000_socket(lua_State *L) {
    int sock_type = luaL_checkint(L, 1);
    if (sock_type != ELUA_NET_SOCK_DGRAM) {
        return luaL_error(L, "Only UDP socket supported.");
    }

    int sock_fd = socket(AF_INET, SOCK_DGRAM, IPPROTO_UDP);

    lua_pushinteger(L, sock_fd);
    return 1;
}
\end{lstlisting}

\vspplus
\lstset{style=lualisting}
$\Rightarrow$ \lstinline{sock = msbiot.cc3000.socket( msbiot.cc3000.SOCK_DGRAM )}
\end{frame}

\section{Ergebnisse}

\begin{frame}[fragile]{Ein nützlicheres Lua-Skript}
\lstset{style=lualisting}
\begin{lstlisting}
local wlan  = msbiot.cc3000

wlan.init("essid", "passwort")

local sock = wlan.socket(wlan.SOCK_DGRAM)
wlan.connect(sock, wlan.packip("196.168.0.3"), 20000)

wlan.send(sock, "Hello, server!")

local data, err = wlan.recv(sock, 1024)
print( string.format('The server sent "%s"', data) )

wlan.close(sock)
wlan.shut_down()
\end{lstlisting}
\end{frame}

\section{Zusammenfassung}

\begin{frame}[t,fragile]{Zusammenfassung}

\vspplus

\begin{columns}[T]
\begin{column}{0.47\textwidth}
\begin{block}{Fazit}
\begin{itemize}
\item Skriptsprachen, MSB-IOT, eLua
\item Portierung, CC1101, CC3000
\end{itemize}    
\end{block}

Grundlage, Orientierung für:

\begin{block}{Agenda}
\begin{itemize}
\item quantitative Auswertung
\item Anpassung, Erweiterung
\end{itemize}    
\end{block}
\end{column}

\begin{column}{0.5\textwidth}
\lstset{style=lualisting, xleftmargin=2em}
\begin{onlyenv}<1>
\begin{lstlisting}
local l = pio.PB_8
local t = 500000

pio.pin.setdir(pio.OUTPUT, l)

while true do
  pio.pin.sethigh(l)
  tmr.delay(tmr.SYS_TIMER, t)
  pio.pin.setlow(l)
end
\end{lstlisting}    
\end{onlyenv}

\begin{onlyenv}<2>
\begin{lstlisting}
local l = pio.PB_8
local t = 500000

pio.pin.setdir(pio.OUTPUT, l)

msbiot.scratch.asymblink()
\end{lstlisting}    
\end{onlyenv}
\end{column}
\end{columns}
\end{frame}

% - Zusammenfassung und Ausblick
%     - LED-Blink-Skript noch mal -- jetzt Taste drücken!
%     - noch viel Arbeit nötig, aber wahrscheinlich lohnend

\end{document}
